\documentclass{article}
\usepackage[utf8]{inputenc}

\title{\textbf{Plano do Projeto}}
\author{\textbf{.PNGenética}}
\date{24 de Março de 2019}

\begin{document}
	
	\maketitle
	
	\section{Nome do Projeto, contexto e objetivos}
	\begin{itemize}
		\item {\textbf{Projeto}}: .PNGenética \newline
	\end{itemize}
	
	Em um cenário de extrapolação dos meios de produção artística, torna-se latente o questionamento quanto aos limites da mesma: o que é arte ? O que a rege ? Qual o requisito a sua produção ? Como definir o que é belo ? Evidentemente algumas dessas perguntas continuam a direcionar a discussão em torno da conceituação do indivíduo, de sua mente e subconsciente de modo a fazer com que a humanidade aproxime-se o máximo do possível de uma resposta. Nesse sentido, em um contexto de desenvolvimento técnico exponencial, arte e tecnologia caminham juntas recriando as formas pelas quais as duas relacionam-se de modo a promover o contato com o indivíduo. \newline
	
	Sendo assim, em conjunto com o grupo "Amudi", segundo a orientação e direcionamento do professor Edson S. Gomi e do monitor João Flesch Fortes, nós, membros do projeto ".PNGenética", decidimos contribuir para o enriquecimento da discussão em torno da relação entre arte e tecnologia. Com esse intuito, fazendo uso de algoritmos genéticos e da plataforma Raspberry Pi, iremos desenvolver ao longo do primeiro semestre de 2019, a criação de imagens, inicialmente abstratas, e disponibilizá-las em site onde as pessoas poderão acessar e votar nas mais interessantes. A partir dessa seleção, as mais "aptas" segundo a ótica dos votantes seguirão para reprodução e mutação, criando, deste modo, uma nova geração que retornará ao processo numa nova iteração.\newline

	O objetivo do projeto ao produzir imagens de acordo com a preferência do inconsciente coletivo é entender melhor quais os estímulos visuais são responsáveis pelo prazer derivado da arte e como se dá a relação entre ser humano e a mesma. Finalmente, tendo sido feita esta análise, as aplicações do projeto podem vir a permear desde os setores produtivos até a educação de modo a promover entre cultura e sociedade ainda na fase de desenvolvimento da criança e do adolescente. 
	
	\section{Escopo do projeto}
	O escopo do projeto incluem e não estão contidos a somente:
	\begin{itemize}
		\item Criação e desenvolvimento de um site por meio do Raspberry Pi.
		
		\item Criação de um algoritmo capaz de gerar imagens de acordo com as preferências do usuário, tendo como base um algoritmo genético. 
		
		\item Desenvolver aplicações práticas do trabalho nos setores da vida em sociedade como a promoção de um site interativo onde setores da indústria que têm seus produtos predominantemente relacionados ao aspecto visual possam realizar enquetes com o usuário de modo a estimular um nicho de produção mais específico.

	\end{itemize}
	
	\section{Cronograma}
	
	O cronograma de desenvolvimento do projeto proposto, segundo o plano inicial, está disponibilizado segundo a tabela abaixo: 
	
	\begin{center}
		\begin{tabular}{| l | l |}
			\hline
			Semana & Objetivos \\ \hline
			Abril 01 (01/04) & Desenvolvimento do site.\\ \hline
			Abril 02 (08/04) & Criação do algoritmo genético.\\ \hline
			Abril 03 (15/04) & Aprimoramento do algoritmo desenvolvido.\\ \hline
			Abril 04 (22/04)& Fim da criação do algoritmo.\\ \hline
			Abril 05 (29/04) & Criação de servidor utilizando Raspberry Pi.\\ \hline
			Maio 01 (06/05)&  Aprimoramento da interação Rasp - indivíduo.\\ \hline
			Maio 02 (13/05)& Início da fase de testes.\\ \hline
			Maio 03 (20/05)& Desenvolvimento da apresentação do projeto.\\ \hline
			Maio 04 (27/05)& Fim da fase de testes e apresentação formal do projeto. \\ \hline
		\end{tabular}
	\end{center}

	
	\section{Custos}
	Os custos previstos são:
	\begin{itemize}
		\item Compra de um domínio caso o hosting via servidores OpenSource não sejam condizentes com a demanda do projeto.
		\item Aulas com duração de três horas semanas juntamente com período de dedicação semanal de cinco horas.
	\end{itemize}

	
	
	\section{Riscos}
	Os riscos ao não cumprimento dos objetivos estipulados ao longo do desenvolvimento do projeto são: \newline
	\begin{itemize} 
		\item A perda do prazo de entrega do trabalho.
		\item A sobrecarga das funções do Raspberry Pi com os elementos do projeto: inicialmente, algoritmo e servidor.
	\end{itemize}

	\section{Comunicação}
	A comunicação entre os integrantes do grupo se dará por meio das seguintes vias:
	\begin{itemize}
		\item Aplicativo WhatsApp.
		\item Repositório Github (https://github.com/lcupertino/GAProj).
		\item Reuniões semanais presenciais com o grupo Amudi e via email.
	\end{itemize}
	
	\section{Integrantes}
	Os alunos do curso de Engenharia de Computação que irão compor o projeto ".PNGenética" serão:	
	\begin{itemize}
		\item Bernardo Coutinho
		\item Joás Barbosa
		\item Lucas Cupertino
		\item Stephanie Miho
	\end{itemize}

\end{document}
